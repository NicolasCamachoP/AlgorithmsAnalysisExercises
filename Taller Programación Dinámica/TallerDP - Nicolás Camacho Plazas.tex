%% LyX 2.3.2 created this file.  For more info, see http://www.lyx.org/.
%% Do not edit unless you really know what you are doing.
\documentclass[spanish]{article}
\usepackage[T1]{fontenc}
\usepackage[latin9]{inputenc}
\usepackage{float}
\usepackage{amsmath}
\usepackage{amsthm}
\usepackage{amssymb}
\PassOptionsToPackage{normalem}{ulem}
\usepackage{ulem}

\makeatletter

%%%%%%%%%%%%%%%%%%%%%%%%%%%%%% LyX specific LaTeX commands.
\newcommand{\noun}[1]{\textsc{#1}}
\floatstyle{ruled}
\newfloat{algorithm}{tbp}{loa}
\providecommand{\algorithmname}{Algoritmo}
\floatname{algorithm}{\protect\algorithmname}

%%%%%%%%%%%%%%%%%%%%%%%%%%%%%% Textclass specific LaTeX commands.
\numberwithin{equation}{section}
\numberwithin{figure}{section}
\theoremstyle{definition}
\newtheorem*{defn*}{\protect\definitionname}

%%%%%%%%%%%%%%%%%%%%%%%%%%%%%% User specified LaTeX commands.
\usepackage{xmpmulti}
\usepackage{algorithm,algpseudocode}
\usepackage{mathtools}
\DeclarePairedDelimiter\floor{\lfloor}{\rfloor}

\makeatother

\usepackage{babel}
\addto\shorthandsspanish{\spanishdeactivate{~<>}}

\providecommand{\definitionname}{Definición}

\begin{document}
	\title{Taller 2 - Programaci�n Din�mica}
	\author{Nicol�s Camacho-Plazas}
	\date{\today}
	\maketitle
	\part{An�lisis del problema}
	El problema requiere que se calculen las particiones binarias cuya diferencia de sumas se m�nima dado un conjunto de n�meros naturales. El conjunto de n�meros naturales se definir� como una secuencia $S$, donde:
	\[
	S=\left\langle s_{1},s_{2},\cdots,s_{n}\right\rangle =\left\langle s_{i}\in\mathbb{N}\right\rangle 
	\]
	Las particiones binarias son dos subconjuntos ($A$ y $B$) formados a partir de los elementos de $S$ y cuya uni�n es equivalente al conjunto $S$, es decir, que todos los elementos de $S$ pertenecer�n a $A$ o a $B$ sin pertenecer a los dos; adem�s, si $s$ es la cantidad de elementos en $A$ y $m$ es la cantidad de elementos de $B$, entonces:
	\[
	\sum_{i=1}^{s}A_{i} = a
	\]
	\[
	\sum_{i=1}^{m}B_{i} = b
	\]
	\[
	n = a + b
	\]
	\[
	x = |a - b|
	\]
	Donde $x$ es el m�nimo valor posible de todas las particiones binarias que se pueden formar a partir de la secuencia $S$.
	\part{Dise�o del algoritmo}
	\section{Entradas}
	\begin{itemize}
		\item Una secuencia $S=\left\langle s_{1},s_{2},\cdots,s_{n}\right\rangle =\left\langle s_{i}\in\mathbb{N}\right\rangle$
	\end{itemize}
	\section{Salidas}
	\begin{itemize}
		\item Los subconjuntos disjuntos $A$ y $B$ formados a partir de $S$ cuya uni�n es equivalente a $S$ y cuyo valor $x$ es el m�nimo posible, siendo $x$ la diferencia entre la suma de todos los elementos del subconjunto $A$ y la suma de los elementos de $B$.
	\end{itemize}
	\part{Algoritmo}
	\begin{algorithm}[H]
		\begin{algorithmic}[1]
			\Procedure{binPartitionBT}{S}
				\State total = 0
				\For{i < |S|}
					\State total = total + S[i]
				\EndFor
				
				\State \textbf{Suponga que $M$ es una matriz booleana de tama�o $(|S|+1)x(total + 1)$}
				
				\For{$i \leq |S|$}
					\For{$j \leq total$}
						\If{j == 0}
							\State M[i][j] = true
						\Else
							\State M[i][j] = false
						\EndIf
					\EndFor
				\EndFor
				
				\State \textbf{Suponga que $BT$ es una matriz de enteros de tama�o $(|S|+1)x(total + 1)x(total)$}				
				
				\For{i = 1; i < |S| + 1}
					\For{j = 1; j < total + 1}
						\If{(j - S[i - 1]) == true}
							\If{M[i - 1][j - S[i - 1]] == true}
								\State M[i][j] = true
								\State BT[i][j] = BT[i][j] \textbf{ Union } BT[i - 1][j - S[i - 1]]
								\State BT[i][j] \textbf{ Union } S[i - 1]
							\ElsIf{M[i - 1][j] == true}
								\State BT[i][j] = true
								\State BT[i][j] = BT[i][j] \textbf{ Union } BT[i - 1][j]
							\EndIf
						\Else
							\State M[i][j] = M[i - 1][j]
							\State BT[i][j] = BT[i][j] \textbf{ Union } BT[i - 1][j]
						\EndIf
					\EndFor
				\EndFor
				
				\State suma = total/2
				\State found = false
				
				\While{$suma\geq 0 \textbf{ and } found == false$}
					\If{M[|S|][suma] == true}
						\State found = true
					\Else
						\State suma = suma - 1
					\EndIf
				\EndWhile
				
				\State secuenciaA = BT[|S|][suma]
				\State secuenciaB = S\textbf{ y no }secuenciaA
			
				\State\Return [secuenciaA, secuenciaB]
			\EndProcedure
		\end{algorithmic}
	\caption{Partici�n binaria con m�nima diferencia de sumas totales}
	\end{algorithm}
	\part{Complejidad}
	Por inspecci�n de c�digo se concluye que la complejidad tienen un orden de $O(nm)$, donde $n = |S|$ y $m = \sum_{0}^{i}S_{i}$.
	\part{Invariante}
	\begin{itemize}
		\item La tabla de memoizaci�n M se llena correctamente, es decir, si $M[i][j] = true$, se puede alcanzar el valor j con los elementos pertenecientes a $\left\langle S_{0}, S_{1}, S_{2}, ...., S_{i-1}\right\rangle$ 
	\end{itemize}

\end{document}




