%% LyX 2.3.2 created this file.  For more info, see http://www.lyx.org/.
%% Do not edit unless you really know what you are doing.
\documentclass[spanish]{article}
\usepackage[T1]{fontenc}
\usepackage[latin9]{inputenc}
\usepackage{float}
\usepackage{amsmath}
\usepackage{amsthm}
\usepackage{amssymb}
\PassOptionsToPackage{normalem}{ulem}
\usepackage{ulem}

\makeatletter

%%%%%%%%%%%%%%%%%%%%%%%%%%%%%% LyX specific LaTeX commands.
\newcommand{\noun}[1]{\textsc{#1}}
\floatstyle{ruled}
\newfloat{algorithm}{tbp}{loa}
\providecommand{\algorithmname}{Algoritmo}
\floatname{algorithm}{\protect\algorithmname}

%%%%%%%%%%%%%%%%%%%%%%%%%%%%%% Textclass specific LaTeX commands.
\numberwithin{equation}{section}
\numberwithin{figure}{section}
\theoremstyle{definition}
\newtheorem*{defn*}{\protect\definitionname}

%%%%%%%%%%%%%%%%%%%%%%%%%%%%%% User specified LaTeX commands.
\usepackage{xmpmulti}
\usepackage{algorithm,algpseudocode}
\usepackage{mathtools}
\DeclarePairedDelimiter\floor{\lfloor}{\rfloor}

\makeatother

\usepackage{babel}
\addto\shorthandsspanish{\spanishdeactivate{~<>}}

\providecommand{\definitionname}{Definición}

\begin{document}
\title{Soluci�n Parcial 1 -  An�lisis Algoritmos}
\author{Nicol�s Camacho-Plazas}
\date{\today}
\maketitle
\part{Algoritmos de promedio}
\section{Dise�o del problema}
\subsection{Entradas}
\begin{itemize}
	\item Una secuencia $
	A=\left\langle a_{1},a_{2},\cdots,a_{n}\right\rangle =\left\langle a_{i}\in\mathbb{T}\right\rangle$, si $n = |A|$, donde 
	el conjunto $\mathbb{T}$ est� conformados por los elementos que tienen definidas las operaciones de suma y resta.
	\item Dos indices $i$ y $j$, donde $i \land J \in [1,...,|A|]$, que representan los l�mites de la subsecci�n de la secuencia de donde se pretende encontrar el promedio.
\end{itemize}
\subsection{Salidas}
\begin{itemize}
	\item Un valor $z=\frac{1}{|A|}\sum_{1}^{|A|}a_{i}$ que por ende contiene el promedio de los valores de la subsecci�n delimitada entre $i$ y $j$.
\end{itemize}
\section{Complejidad}
\subsection{Algoritmo 1 - Promedio inocente}
Al realizar inspecci�n de c�digo, por el �nico for, se concluye que el algoritmo tiene un orden de complejidad de $O\left(\left|A\right|\right)$.
\subsection{Algoritmo 2 - Promedio dividir y vencer}
De acuerdo al teorema maestro, y teniendo en cuenta que el problema divide en dos los datos, usa dos veces la recursi�n y la complejidad fuera de la recursi�n es $O(1)$, se puede determinar que:
$$
T(n)=\begin{cases}
O(1),
& \mbox{caso base $n<=1$,}\\
2T(\frac{n}{2})+O(1), & \mbox{si $n\in|A|$}
\end{cases}
$$
De modo que al estudiar el caso uno del teorema maestro, se establece que:
$O(1) = O(n^{\log_2 2-\epsilon})$

$\therefore {\log_2 2-\epsilon} = 0 \land \epsilon = 1$

$\therefore T(n)\in\theta(n)$
\section{Invariante}
\subsection{Promedio inocente}
$\mu$ siempre va a ser el total de la secuencia hasta $A[k]$ donde $k\in[1,...,|A|]$.
\subsection{Dividir y conquistar}
\begin{itemize}
	\item $x$ siempre contiene el promedio parcial de la secuencia contenida entre $i$ y $q-1$.
	\item $y$ siempre contiene el promedio parcial de la secuencia contenida entre $q+1$ y $j$.
\end{itemize}
\subsection{De acuerdo a los puntos anteriores, ?cu�l algoritmo es mejor de implementar en un contexto d�nde se deba calcular muchos promedios en conjuntos de datos de tama�os del orden de los billones de elementos?}
Teniendo en cuenta que una operaci�n de divisi�n es m�s compleja de realizar que una suma o resta y el algoritmo inocente solo realiza una por promedio, la pila de ejecuciones que genera una recursi�n tan extensa, y la cantidad de variables empleadas en la recursi�n, se concluye que resulta m�s conveniente utilizar el algoritmo de "Promedio inocente".
\part{Escritura de un algoritmo}
\section{An�lisis del problema}
Dada una secuencia $A$ de elementos $A_i$ donde $A_i\in\mathbb{T}$ e $i\in [1,...,|A|]$, cuyo
conjunto $\mathbb{T}$ est� conformado por elementos con los que es posible utilizar la operaci�n "$<$", y unos l�mites $i$ y $j$ que cumplen $i\land j\in [1,...,|A|]\land i<j$, se busca encontrar la cantidad y l�mites de los rangos aditivos presentes en dicha secuencia. Un rango aditivo 
se define como una pareja de �ndices $[a,b]$ donde $A[a]>A[i] \land A[a]<A[j]$ y $A[b]>A[i] \land A[b]<A[j]$.
\section{Dise�o del Algoritmo}
\subsection{Entradas}
\begin{itemize}
	\item Una secuencia $A$ descrita en el an�lisis del problema y de donde se quieren.
	\item Un indice $i$ que representa el valor menor de la pareja para encontrar los rangos aditivos y donde 
	$i\in [1,...,j-1]$.
	\item Un indice $j>i$ que representa el valor mayor de la pareja para encontrar los rangos aditivos y donde 
	$j\in [n,...,|A|]$ donde $n>i$.
\end{itemize}
\subsection{Salidas}
\begin{itemize}
	\item Un valor $z$ que representa la cantidad de rangos aditivos que cumplen con la pareja ${i,j}$ dentro de la secuencia $A$.
\end{itemize}
\section{Algoritmo}
Para resolver el algoritmo se utilizan dos algoritmos, <<CountAditiveRangesAux>> en el que se realiza la recursi�n siguiendo el paradigma de dividir y vencer, y <<CountAditiveRanges>> el cual adapta los par�metros del usuario y adec�a el retorno. 
\begin{algorithm}[H]
	\begin{algorithmic}[1]
		\Procedure{CountAditiveRangesAux}{$A$,$b$, $e$, $range$}
		
			\If{$b=e$}
				\If{$range[1]<A[b]\textbf{ and }A[b]<range[2]$}
					\State\Return 1
				\Else
					\State\Return 0
				\EndIf
			\Else
				\State $q=\lfloor(e+b)/2\rfloor$
				\State $totalLeft = CountAditiveRangesAux(A,b,q,range)$
				\State $totalRight = CountAditiveRangesAux(A,q+1,e,range)$
				\State\Return $totalLeft+totalRight$
			\EndIf
		\EndProcedure
	\end{algorithmic}
	\caption{N�mero posibles l�mites de rangos aditivos - Dividir y vencer}
\end{algorithm}
El siguiente algoritmo utiliza un m�todo llamado \textit{Combinatorial} que puede ser cualquier m�todo 
de cualquier librer�a que calcule la combinatoria de dos n�meros enteros.
\begin{algorithm}[H]
	\begin{algorithmic}[1]
		\Procedure{CountAditiveRanges}{$A$, $i$, $j$}
			\State $total = CountAditiveRangesAux(A,1,|A|,[i,j])$
			\State\Return $\frac{total!}{2(total-2)!}+total$
		\EndProcedure
	\end{algorithmic}
	\caption{N�mero de rangos aditivos en la secuencia}
\end{algorithm}
\section{An�lisis de complejidad}
Al utilizar el teorema maestro, teniendo en cuenta que se divide en 2 el problema y se realizan dos llamados recursivos, se establece que:
$$
T(n)=\begin{cases}
O(1),
& \mbox{caso base $n=1$,}\\
2T(\frac{n}{2})+O(1), & \mbox{si $n\in|A|$}
\end{cases}
$$
De modo que al estudiar el caso uno del teorema maestro, se establece que:
$O(1) = O(n^{\log_2 2-\epsilon})$

$\therefore {\log_2 2-\epsilon} = 0 \land \epsilon = 1$

$\therefore T(n)\in\theta(n)$
\section{Invariante}
\begin{itemize}
	\item \textit{CountAditiveRangesAux}: <<$totalLeft$>> siempre va a tener el total de los elementos que est�n dentro de el rango comprendido entre $i$ y $j$ de la subsecuencia delimitada por $b$ y $q$. <<$totalRight$>> siempre va a tener el total de los elementos que est�n dentro de el rango comprendido entre $i$ y $j$ de la subsecuencia delimitada por $q+1$ y $e$.
	\item \textit{CountAditiveRanges}: <<$total$>> contiende la cantidad de elementos que pueden representar 
	los l�mites de los rangos aditivos. Su combinatoria contempla la cantidad de rangos que se pueden formar 
	entre dichos elementos y al sumar el total se contemplan los casos en los que el rango comience y termine por el mismo elemento. 
\end{itemize}
\end{document}



